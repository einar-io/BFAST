\section{Einar's part}

\begin{frame}[fragile]{Overview}
\begin{enumerate}
    \item \texttt{bfast\_step\_5}
    \item \texttt{bfast\_step\_6} Aske Dorge's rule of thumb of shared
        memory.
    \item ..
\end{enumerate}
    \pause
    I prefer if you would ask questions during the presentation.
\end{frame}


\section{\texttt{bfast\_step\_5}} % Einar

% Et af de argumenter, der bruges til at støtte udviklingen af autonome
% våbensystemer, påpeger at autonome våbensystemer vil være i stand til at undgå
% nogle af de fejl, mennesker begår, og derfor vil autonome våbensystemer på
% sigt være med til at nedbringe antallet af civile ofre i krig. Beskriv Peter
% Asaros holdning til argumentet og diskuter, om I selv finder argumentet
% holdbart.

\begin{frame}[fragile]{Futhark}

%  -- Cosmin's tip for this bit: Assume N < 1024
%
%  ---------------------------------------------
%  -- 5. filter etc.                          --
%  ---------------------------------------------
%  -- Nss:        Nss[i] where 0<=i<m is the number of valid (non-NaN) entries
%  --             for time series i
%  -- y_errors:   y_errors[i] where 0<=i<m is an array of error values for time
%  --             series i, partitioned such that valid (non-NaN) entries come
%  --             before invalid (NaN) entries.
%  -- vald_indss: vald_indss[i] where 0<=i<m is an array of indices indicating
%  --             the original positions of the elements in y_errors[i], i.e.,
%  --             before partitioning.
\begin{minted}{haskell}
let (Nss, y_errors, val_indss) =
unsafe ( intrinsics.opaque <| unzip3 <|
  map2 (\y y_pred ->
    let y_error_all = zip y y_pred |>
    F
      map (\(ye,yep) -> if !(f32.isnan ye) 
                        then ye-yep 
                        else f32.nan )
    let (tups, Ns) = zip2 y_error_all (iota N) |>
      partitionCos (\(ye,_) -> !(f32.isnan ye)) (0.0, 0)
    let (y_error, val_inds) = unzip tups
    in  (Ns, y_error, val_inds)
      ) images y_preds )
\end{minted}

\end{frame}

\begin{frame}[fragile]{CUDA}

\begin{minted}{cuda}
void bfast_step_5_run(struct bfast_state *s)
{
  /* .. */
  dim3 block(N, 1, 1);
  dim3 grid(m, 1, 1);
  const size_t shared_size = N * sizeof(int);
  bfast_step_5<<<grid, block, shared_size>>>
    (d_Y, d_y_preds, d_Nss, d_y_errors,
    d_val_indss, N);
\end{minted}

\end{frame}

\section{\texttt{bfast\_step\_6}} 


\begin{frame}[fragile]{CUDA}

\begin{minted}{cuda}
__global__ void bfast_step_6_reuse(float *Yh, float *y_errors, int *nss,
    float *sigmas, int n, int N, int k2p2)
{
  /* .. */
  __shared__ int num_valids[1024];
  num_valids[threadIdx.x] = !isnan(yh[threadIdx.x]);
  __syncthreads();
  scaninc_block_add<int>(num_valids);
  int ns = num_valids[n - 1];
  __syncthreads(); // necessary because shared memory is reused

  //sizeof{int} == sizeof{float}
  float *sigma_shared = (float *) num_valids; // Here be dragons
  float val = threadIdx.x < ns ? y_error[threadIdx.x] : 0.0;
  val = val * val;
  sigma_shared[threadIdx.x] = val;
  __syncthreads();
  scaninc_block_add<float>(sigma_shared);

  if (threadIdx.x == 0) {
    sigmas[blockIdx.x] =
      __fsqrt_rd(sigma_shared[n - 1] / ((float)(ns - k2p2)));
    nss[blockIdx.x] = ns;
  }
}
\end{minted}
\pause
\mint{cuda}|sizeof{int} == sizeof{float}|

\end{frame}

\begin{frame}[fragile]{Launch parameters}
\begin{minted}{cuda}
void bfast_step_6_reuse_run(struct bfast_state *s)
{
  /* .. */
  dim3 grid(m, 1, 1);
  dim3 block(n, 1, 1);
  bfast_step_6_reuse<<<grid, block>>>
    (d_Y, d_y_errors, d_nss, d_sigmas, n, N, k2p2);
}

\end{minted}
\end{frame}



