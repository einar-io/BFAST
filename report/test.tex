\section{Testing and code structure}
In this section, we briefly explain the structure and some of the design
choices in our source code.
The goal of this section is to make the source code easier to navigate for the
reader, and to briefly explain how to run benchmarks, perform tests, and
generate test datasets.

The handed-in zip file contains two folders, \texttt{data} and \texttt{src}.
A brief overview of the contents of the \texttt{src} folder is as follows:
\begin{description}
\item[\texttt{main.c}:]
  Entry point for BFAST and parsing of command line options.
\item[\texttt{bfast\_util.cu}, \texttt{bfast\_util.cu.h}, and \texttt{bfast.h}:]
  Utilities for executing full BFAST runs, and for creating and running tests
  for individual components of the BFAST algorithm.
\item[\texttt{kernels/} directory:]
  Kernels for executing BFAST, wrapper host functions for invoking them with
  the correct grid and block sizes, test declarations for every kernel.
\item[\texttt{bfast.cu}:]
  Functions for executing full BFAST runs.
  A full BFAST run is declared as an array of \texttt{struct bfast\_step},
  which is passed to a utility function from \texttt{bfast\_util.cu} that will
  execute the specified kernels.
\item[\texttt{timer.h}:]
  Utility functions for measuring runtimes.
\item[\texttt{panic.h}:]
  From the Futhark repository, ISC license.
  Contains helper functions for panicking and formatting messages.
  Minor BFAST-related changes made.
\item[\texttt{values.c} and \texttt{values.h}:]
  From the Futhark repository, ISC license.
  Contains a parser for binary and string formatted Futhark values.
\item[\texttt{memcheck-runner.sh}:]
  Wrapper script for calling the \texttt{bfast} executable.
  Used with the \texttt{test-memcheck} make target.
\item[\texttt{Makefile}:]
  Makefile used for building bfast, running tests and benchmarks, and
  generating datasets.
\item[\texttt{fut/} directory:]
  Contains \texttt{bfast-kernels.fut}, another Futhark version of the BFAST
  algorithm, in which each step of the algorithm is put into its own entry
  point.
  This file also contains functions for generating test datasets.
  Also contains \texttt{generate-datasets.sh}, a shell script for compiling and
  executing the \texttt{bfast-kernels.fut} program to generate test datasets.
\item[\texttt{fut-handout/} directory:]
  Handed out implementations of BFAST.
\item[\texttt{tests/} directory:]
  Contains \texttt{tests.fut}, an empty Futhark program only containing test
  cases.
  These test cases are run on our compiled CUDA implementation of BFAST by
  giving special parameters to the Futhark test program, \texttt{futhark-test}.
\end{description}

\subsection{Make targets}
\begin{description}
\item[\texttt{all}:] Compile BFAST
\item[\texttt{debug}:] Compile BFAST with various debug options
\item[\texttt{test}:] Test the CUDA implementation using \texttt{futhark-test}.
\item[\texttt{test-memcheck}:]
Test the CUDA implementation using \texttt{futhark-test} while checking for
out-of-bounds memory accesses with \texttt{cuda-memcheck}.
\item[\texttt{benchmark}:]
Benchmark the \texttt{bfast-opt} variant of the full BFAST algorithm.
\item[\texttt{bennchmark-alt}:]
Benchmark the \texttt{bfast-opt-alt} variant of the full BFAST algorithm.
\item[\texttt{benchmark-naive}:]
Benchmark the \texttt{bfast-naive} variant of the full BFAST algorithm.
\item[\texttt{datasets}:]
Generate datasets for testing.
Must be executed once before running the \texttt{test} or
\texttt{test-memcheck} targets.
\end{description}
The test-related targets may take some time to execute.


\subsection{Regarding the declaration of BFAST algorithms}
In \texttt{src/bfast.cu}, as already mentioned, several variants of full BFAST
runs are defined using arrays of \texttt{struct bfast\_step}.
Each \texttt{struct bfast\_step} denotes an action that should be taken: either
a call to a specified kernel, or a transpose, or an \textit{untranspose}.
Clearly, it does not make sense to talk about an untranpose operation
mathematically, since the transpose operation is its own inverse, but in our
case, the implementation details of \texttt{struct bfast\_state} and related
structs and functions make it necessary to distinguish between, for example,
going from \(Y\) to \(Y^T\) and going from \(Y^T\) to \(Y\).


