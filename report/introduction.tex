\section{Introduction}
% Keywords:
% Semantic preserving
% Interpretations of values.

\subsection{The \textsc{BFAST} algorithm}
\subsubsection{The \textsc{BFAST} article}
The project is based on the algorithm described in \cite{bfast}.
More precisely it is a translation of the implementation written in Futhark and
handed out as \texttt{bfast-distrib.fut}.

\subsubsection{Purpose of \textsc{BFAST}}
The algorithm takes as input, among other things, \texttt{m} time series of
length \texttt{N}.
Out of the \texttt{N} entries in each time series, there is a historic period
of length \texttt{n}, based on which predictions are made for the the remainder
of the period (of length \texttt{N-n}).
This period is called the monitor period.

The output of the algorithm is an array of length \texttt{N-n}, in which the
value of each element indicates the extent to which the predictions for the
monitor period deviate from the observed values of the monitor period.
High values in the output represent unexpected changes and are called
\enquote{breaks}.

The algorithm accounts for seasonal trends in the time series, and it takes
into account the precense of \texttt{NaN} values in the datasets.

We will not be concerning ourselves with further interpretation of the
semantics, as the focus of this project is on semantics-preserving code
optimizations and their impact on runtime.


\subsubsection{Dataset and variables}

The dataset \texttt{sahara} was supplied for testing. Certain quantities
describing the dataset input and output are listed in \autoref{tbl:scalars}.


\begin{figure}[H]
    \centering
    \begin{tabular}{l l r r r}
    \textbf{Variable} & \textbf{Interpretation}             & \textbf{\texttt{sahara} dataset} & \textbf{Upper bound} \\ \hline
        n                & length of historical period  & 228                     & N-1 \\
        N                & length of full period        & 414                     & 1024   \\
        N-n              & length of monitor period     & 186                     & 1024 \\
        k                &                              & 3                       & 3 \\
        k2p2             & \(k\times2 + 2\)             & 8                       & 8 \\
        frec             & images per year              & 12.0                    & \\
        m                & number of pixels             & 67968                   & \\
        hfrac            & Moving sum bandwidth \(\approx \nicefrac{h}{n}\)    & 0.25                    & \\
        lam              &                              & 1.736126                &
    \end{tabular}
    \caption{List of input scalar variables, their assumed upper bound when known, and values for the \texttt{sahara} dataset.}
    \label{tbl:scalars}
\end{figure}



\subsubsection{Benchmark machines}\label{sec:machines}

The following machines have been used for benchmarking:
%
\begin{description}
  \item[Machine 1 (M1): ]
    \texttt{gpu03-diku-apl.science.ku.dk}, GPU: GeForce GTX 780 Ti, CPU: Intel Xeon E5-2650 v2
  \item[Machine 2 (M2): ]
    Jakob's desktop PC, GPU: GeForce GTX 970, CPU: Intel Core i7-4790K
\end{description}

Unless stated otherwise we run our benchmarks on Machine 1.

