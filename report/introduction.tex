\section{Introduction}
% Semantic preserving
% Interpretations of values.

\subsection{\textsc{BFAST} article}
The project is based on \cite{bfast}.

Break detection

\texttt{nan}.

interpretation of


\subsection{Non-optimization}
While not an optimization, this is still an in important learning point for us,
that gave us a speed up from roughly 36ms to 26ms.
We will not concern us with further interpretation, as the focus is on
semantic preserving code optimization.


\subsection{Thread guards??}
Not an optimization Dynamic block size vs worst-case
Despite we do this we still include a \texttt{return} statement in the begiining
of each kernel in order as an sanity check. Ideally no threads should be spawned
just to return immediately. This does happen for some kernels, when the block
size does not divide the size of the dataset, and the last block more threads
than work.

\subsection{Rowmajor form and macros}

The matrices are dense and stored in a row-major order. This means that all elements
of row \(n\) is stored before elements of row \(n+1\). Naturally all elements
within a row is stored in the order of the their corresponding column.


\begin{figure}[H]
    \centering
    \ecuda[firstline=19,lastline=20]{../src/bfast_util.cu.h}
    \caption{Macros}
    \label{cuda:macros}
\end{figure}


\begin{description}
    \item[\texttt{IDX\_2D(i,j,columns)}] The two macros in \autoref{cuda:marcros} are used to
calculate the offset into a two-dimentional array when only the start address is
passed as in the case for passing a pointer to an array of arrays in C-like languages.

\item[\texttt{CEIL\_DIV(x,y)}] Ceiled integer division is used to calculate
\(\ceil*{x/y}\) without using floating-point arithmetic assuming both \(x, y \in
\mathbb{N}^{+} \).

\end{description}

    

\subsection{Variables and their bounds}

\begin{figure}[H]
    \centering
    \begin{tabular}{l l r r r}
        Variable & Interpretation              & Lower bound & \texttt{sahara} dataset & Upper bound \\ \hline
        n        & length of historical period & 1           & 228                     & N-1 \\
        N        & length of full period       &             & 414                     & 1024   \\
        N-n      & length of monitor period    &             & 186                     & \\
        k        &                             &             & 3                       & \\
        k2p2     & k\times2                    &             & 8                       & \\
        frec     & images per                  &             &                         & 12.0  \\
        m        & number of pixels            &             &                         & 67968 \\
        hfrac    &                             &             & 0.25                    & \\
        lam      &                             &             & 1.736126                & 
    \end{tabular}
    \caption{List of input scalar variables, their bounds, and values for the
    \texttt{sahara} dataset.}
    \label{tbl:scalars}
\end{figure}








