\subsection{
    Step 1: Generating X
}

\begin{figure}[H]
    \centering
    \ecuda[firstline=7,lastline=36]{../src/kernels/bfast_others.cu}
    \caption{CUDA kernel for generating X.}
    \label{cuda:kernel1}
\end{figure}



The first CUDA kernel in \autoref{cuda:kernel1} is concerned with line 103---104
in the Furthark code, seen in \autoref{fut:kernel1} and defined in \autoref{fut:mkX},
which generates the elements in the matrix \(X\).
% corresponding to \(X\) on line
% 1 in \cite[Algorithm 1, \textsc{BFAST}, p. 3]{bfast}.

The remainder of \autoref{fut:kernel1} is handled differently in CUDA. Instead of
defining a new variable to hold an initiating range of existing values, we pass a
the index, at which the subrange ends. This allows for less copying of data.

Looking further at \autoref{cuda:kernel1}, we see that the kernel does not
return any values, i.e. it is type \texttt{void}, so the output matrix is
written to the location passed in trough a pointer in the formal parameters.

%This is allows for the orchestrating code to do the allocation and better reuse
%device allocated memory between kernels.

In line 29 and 31 we use CUDA's intrinsic trignometric functions which,
according to NVIDIA:
\enquote{
    \textit{
Calculate the fast approximate sine of the input argument.
}
}

Benchmarks of \texttt{kernel1} show that the intrinsic functions result in a
running time of around 14-16 ms which is the same as using the standard
\texttt{sinf} and \texttt{cosf} functions. This is in line with our
understanding that that the intrinsic functions are substituted for the standard
ones by the compiler. It seems like a reasonable approach to use them where
possible and we can safely do so, because we know that our code still validates.

\begin{figure}[H]
    \centering
    \ehaskell[firstline=100,lastline=104]{../src/fut-handout/bfast-distrib.fut}
    \caption{Futhark function for generating X.}
    \label{fut:kernel1}
\end{figure}


\begin{figure}[H]
    \centering
    \ehaskell[firstline=29,lastline=40]{../src/fut-handout/bfast-distrib.fut}
    \caption{Futhark function for generating X.}
    \label{fut:mkX}
\end{figure}



