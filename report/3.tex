\section{
    Step 3: Inverting \texorpdfstring{\(X\)}{X}
}

The kernel for matrix inversion is seen in \autoref{cuda:kernel3}. It is based
on the Futhark function seen in \autoref{fut:kernel3} as well as the helper
functions \autoref{fut:gaussjordan} and \autoref{fut:matinv}.

The algorithm seems follow a generalized version the procedure taught in a basic
linear algebra course:

\begin{enumerate}
    \item Juxtapose the identity matrix \(I\) with the matrix to be inverted,
        \(A\), to get \([A|I]\).
    \item Use Gauss-Jordan elimination on \([A|I]\) the in oder to get
        \([I|A^{-1}]\).
    \item Drop the first half containing \(I\) and return \(A^{-1}\).
\end{enumerate}

Step 1 and 3 are done in \autoref{fut:matinv} and corresponds to line 81---91 of
\autoref{cuda:kernel3}. Step 2 is done in line 93---112 of \autoref{fut:gaussjordan}.  
We made the following four considerations during the CUDA convertion:

\paragraph{Block size}
We chose a block size of \(\texttt{k2p2} \times (2)\texttt{k2p2}\). This means
we can load from global memory or assign a value from the identity matrix in one
\texttt{if-else} statement. Please notice that the branches execute
sequentially.

\paragraph{Shared memory}
The result of the above is written into a location in shared memory, which is a
form for scratchpad memory shared among all threads in the block, so we can
access and manipulate it fast in the next step

\paragraph{Block size}
Since we have more sequential steps and these steps manipulate the same memory
location, it is important to use the CUDA API call \texttt{\_\_syncthreads()}.
This effectively creates synchronization point, where all warps wait on all
others in the block, making their view of the memory \textit{coherent}.

\paragraph{Non-invertible matrices}
Like the \texttt{bfast-distrib.fut}, we do not concern us self with the result
of applying the algorithm on a non-invertible matrix. We are satisfied with
preserving the semantics.


\begin{figure}[H]
    \centering
    \ecuda[firstline=58,lastline=113]{../src/kernels/bfast_others.cu}
    \caption{CUDA kernel for inverting X.}
    \label{cuda:kernel3}
\end{figure}


\begin{figure}[H]
    \centering
    \ehaskell[firstline=123,lastline=124]{../src/fut-handout/bfast-distrib.fut}
    \caption{The Futhark calling code for matrix inversion.}
    \label{fut:kernel3}
\end{figure}


\begin{figure}[H]
    \centering
    \ehaskell[firstline=60,lastline=72]{../src/fut-handout/bfast-distrib.fut}
    \caption{The matrix inversion function that constructs the \([A|I]\) matrix
    and returns the latter half upon completion of Gauss-Jordan elimination.}
    \label{fut:matinv}
\end{figure}

\begin{figure}[H]
    \centering
    \ehaskell[firstline=47,lastline=58]{../src/fut-handout/bfast-distrib.fut}
    \caption{The Gauss-Jordan elimination function in \texttt{bfast-distrib.fut}.}
    \label{fut:gaussjordan}
\end{figure}





