% Include futhark code in beginning of each section.
% Describe how our code is translated and diverges from the handed-out code.

\section{Translation and optimization}

Before we look at how each kernel is translated, we mention a couple of
reoccurring traits of our code:

\subsubsection{Guards} In the beginning of each kernel we include a guard, i.e.
an early \texttt{return} statement, as an sanity check. A simple example can be
seen \autoref{cuda:guards}.

\begin{figure}[H]
    \centering
    \ecuda[firstline=204,lastline=204]{../src/kernels/bfast_others.cu}
    \caption{Guard to provide an early exit if thread index is larger than expected.}
    \label{cuda:guards}
\end{figure}

Ideally no threads should be spawned just to return immediately. However, this
does happen for some kernels, for example when the block size does not divide
the size of the dataset, and the last block has more threads than work.

\subsubsection{Dynamic block sizes}
While not an optimization per se, using dynamically calculated values instead
of statically chosen \enquote{worst-case} values for block sizes generally
improves occupancy, and thus, performance.
% a significant reduction in running time from around 36 ms to 24 ms. 

%This alone reduced the running time of the  baseline translation found in
%\texttt{src/bfast-naive.cu} from 50-70 ms to around 26 ms.

\subsubsection{Row-major order and macros}
The matrices are dense and stored in a row-major order.
This means that all elements of a row \(n\) are stored before any element of
row \(n+1\).
Naturally, all elements within a row are stored in the order of the their
corresponding column.

\subsubsection{Helper macros}
\begin{figure}[H]
    \centering
    \ecuda[firstline=19,lastline=20]{../src/bfast_util.cu.h}
    \caption{Macros}
    \label{cuda:macros}
\end{figure}

The two macros in \autoref{cuda:macros} are used throughout the code base to
make certain often-repeated calculations easier and faster to read, and to
minimize the risk of making a mistake while writing the code.

\begin{description}

    \item[\texttt{IDX\_2D(i,j,columns)}] is used to calculate the
        offset into a two-dimensional array for a set of coordinates.

    \item[\texttt{CEIL\_DIV(x,y)}] Ceiled integer division is used to calculate
        \(\ceil*{x/y}\) without using floating-point arithmetic assuming both
        \(x, y \in \mathbb{N}^{+} \).

\end{description}

\include*{transpose} % Jakob
\include*{scan} % Jakob. Further optimization by reduce.
\include*{1} % Einar: nævn brugen af intrinsics. ok
\include*{2} % Jakob
\include*{3} % Einar __syncthreads(). ok
\include*{4} % Einar. pragmas. 
%\include*{4b} % Einar
%\include*{4c} % Einar. Flipped matrix indentity. Refer ti 4a.
\include*{5} % Jakob scan
\include*{6} % Einar. Try without reuse shr mem. ok.
\include*{7a} % Einar. ok
\include*{7b} % Einar. Less copying by using kernel. ok.
\include*{8} % Einar
