\subsection{
    Step 7a: Moving sums first \texttt{MO\_fsts}
}

The moving sums first element is calculated in \autoref{fut:kernel7a}. We have
translated it to CUDA in \autoref{cuda:kernel7a}. 
After the \texttt{scaninc\_block\_add\_nowrite<float>(errs)} on line 294, each
thread hold the result in \texttt{val}. The last thread in the block, \(h-1\), has the sum of
the whole array corresponding to a reduce.

We specify the size of the
shared memory as a launch parameter in \autoref{cuda:kernel7a_run}. 


\begin{figure}[H]
    \centering
    \ehaskell[firstline=181,lastline=188]{../src/fut-handout/bfast-distrib.fut}
    \caption{Futhark function for calculating \texttt{MO\_fsts}.}
    \label{fut:kernel7a}
\end{figure}


\begin{figure}[H]
    \centering
    \ecuda[firstline=275,lastline=299]{../src/kernels/bfast_others.cu}
    \caption{CUDA kernel for calculating  \texttt{MO\_fsts}.}
    \label{cuda:kernel7a}
\end{figure}

\begin{figure}[H]
    \centering
    \ecuda[firstline=301,lastline=312]{../src/kernels/bfast_others.cu}
    \caption{CUDA kernel for calculating  \texttt{MO\_fsts}.}
    \label{cuda:kernel7a_run}
\end{figure}

