\subsection{
    Step 6: \texttt{ns} and \texttt{sigma}
}

Step 6 corresponds to the code seen in \autoref{fut:kernel6}.

\begin{figure}[H]
    \centering
    \ehaskell[firstline=165,lastline=178]{../src/fut-handout/bfast-distrib.fut}
    \caption{Futhark function for calculating \texttt{nss} and \texttt{sigmas}.}
    \label{fut:kernel6}
\end{figure}

In our translation, seen in \autoref{cuda:kernel6},
we reuse shared memory for separate tasks on line 37 by aliasing a pointer to
it. 

Using the rule of thumb attributed to Aske Dorge on \cite[p. 29]{pmphL5}, which
states that:
\enquote{
    \textit{
\approx 8 words of shared memory per thread does not degrade performance
}}, 
we should be okay
with using two arrays of 1024 elements for \(n\approx400\) giving around
\(\nicefrac{2048}{400} = 5\). However, experiment shows that the kernel runtime goes
from 2144.56 µs when reusing the shared memory to 2508.82 µs when allocating a
new array, equaling a slowdown close of 17\%. We reason that this is safe to
do, as the shared memory is never used for more than one purpose at a time.

\begin{figure}[H]
    \centering
    \ecuda[firstline=2,lastline=49]{../src/kernels/bfast_step_6.cu}
    \caption{CUDA kernel for calculating \texttt{nss} and \texttt{sigmas}.}
    \label{cuda:kernel6}
\end{figure}

