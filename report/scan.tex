\subsection{
  Helper kernel: Inclusive block scan with addition
}

Scan is an operator that is often used in parallel programming.
For our BFAST implementation, we only need to be able to perform scans within
thread blocks, meaning on arrays of maximum size 1024.
Additionally, we will only be performing inclusive scans with the
\texttt{(+)}-operator.

The scan implementation used for this project can be found in
\texttt{src/kernels/bfast\_helpers.cu.h}.
The implementation consists of three functions.
The functionality of each of these three is as follows:
\begin{description}
\item[\texttt{scaninc\_warp\_add}]
This function applies an inclusive scan with the \texttt{(+)}-operator to each
slice of 32 contiguous elements of an input array.
This is done by utilizing warp-synchronicity -- the fact that threads in a warp
execute instructions in an SIMD fashion, so no synchronization is needed.
\item[\texttt{scaninc\_block\_add\_nowrite}]
This function performs the full block-level scan on its input array.
In brief, this done by:
\begin{enumerate}
\item
  Applying \texttt{scaninc\_warp\_add} to scan each slice of 32 contiguous
  elements of the input array.
\item
  Scanning using \texttt{scaninc\_warp\_add} over a temporary array containing
  the last element of every 32 element long slice.
  This temporary array will not be longer than 32.
\item
  "Distributing" the result of the scan on the temporary array to the remaining
  elements in the array. %% XXX: "see slides for details"? Lab2_presentation.pdf last slide
\end{enumerate}
It is worth noting that this function does not write the scanned array to the
input buffer, but instead, the return value for any thread with index \(k\) is
the element with index \(k\) in the result array.
This feature is why the name of the function ends with \texttt{nowrite}, and it
is useful for when the scan is being used as a reduce, meaning all values
except the last can be disregarded, or when a map needs to be applied to the
result array immediately after the scan returns. % XXX: skær ud i pap at det her er godt fordi færre hukommelsestilgange?
It is, however, unlikely that using this function in these use cases will have
a significant positive performance impact, since all scans should be done in
shared memory, and accessing this kind of memory is almost as fast as accessing
registers.
\item[\texttt{scaninc\_block\_add}]
This function is a simple wrapper around \texttt{scaninc\_block\_add\_nowrite},
which writes the return values to the input array.
\end{description}


