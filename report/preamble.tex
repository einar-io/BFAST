%Needs to go before document class.  http://tex.stackexchange.com/a/589
\RequirePackage[l2tabu,orthodox]{nag}

% http://tex.stackexchange.com/questions/62311/a4paper-where-should-i-declare-it-in-document-class-or-geometry
% http://tex.stackexchange.com/questions/79046/baselinestretch-vs-setspace
%\documentclass[12pt, a4paper]{article}
\documentclass[a4paper]{article}
%\usepackage[margin=1.8cm]{geometry}
%\usepackage{setspace}
%\setstretch{1.3}
% Required for inputting international characters
%\usepackage[utf8]{inputenc} 

% 2018-02-27
% See fontspec below
% Output font encoding for international characters
%\usepackage[T1]{fontenc} 

% 2018-02-27
% Switching from LATEX to LuaLATEX
% http://dante.ctan.org/tex-archive/info/luatex/lualatex-doc/lualatex-doc.pdf
% Don't load fontenc nor textcomp, but load fontspec.
% The fontspec package Font selection for XeLaTeX and LuaLaTeX
% Don't use any package that changes the fonts, but use fontspec's commands instead.
\usepackage{fontspec} 
\defaultfontfeatures{Ligatures=TeX}
\usepackage{lualatex-math}


% Package for extended math settings, e.g. \eqref
\usepackage{amsmath}
\usepackage{mathtools}
\usepackage{amssymb}
\usepackage{amsthm}
\usepackage{unicode-math} % ???
%\setmainfont{Linux Libertine O}
%\setsansfont{Linux Biolinum O}
%\setmonofont{Mononoki}[Scale=MatchLowercase]
\setmathfont{texgyrepagella-math.otf}

% Other fonts
%\usepackage{mathpazo} % Palatino as regular font
%\usepackage[euler-digits,small]{eulervm} % Euler as virtual math font. Euler for math and numbers
\AtBeginDocument{\renewcommand{\hbar}{\hslash}}
%\usepackage{sourcecodepro} % Source Code Pro as \ttfamily
%\usepackage{garamondx} % Garamond as regular font
%\usepackage{cabin} % Cabin as sans-serif
%\usepackage[varqu,varl]{zi4} % inconsolata as monospace
%\rmfamily <-- regular font
%\sffamily <-- sans-serif
%\ttfamily <-- monospace

% Extended language rules
%\usepackage[english]{babel} 
%\usepackage[main=english]{babel} 
% 2018-02-27
% polyglossia is a modern replacement for use with LuaLatex
% https://tex.stackexchange.com/a/371541
%\DeclareLanguageMapping{english}{british}
\usepackage{polyglossia}
%\setdefaultlanguage{english}
\setmainlanguage[variant=british]{english}
%\setotherlanguage{danish}
%\setotherlanguage{german}


% Force text to be left-justified
\usepackage[document]{ragged2e} 

% Make small adjustments to the font-size for prettier results
\usepackage{microtype} 

% Package for inserting clickable hyperlinks in pdf versions as produced by pdflatex
\usepackage{hyperref}
% https://tex.stackexchange.com/a/207745
\newcommand*{\Appendixautorefname}{Appendix}


\usepackage{xcolor}
\hypersetup{
    colorlinks,
    linkcolor={red!50!black},
    citecolor={blue!50!black},
    urlcolor={blue!80!black}
}
\usepackage[all]{hypcap} % makes it so the hyperref link scrolls to the figure instead of the caption

% Package for including figures. TeX and thus LaTeX was developped before the existence of directory file-structures, but the graphicspath let's you add directories, that the \includegraphics will search.
\usepackage{graphicx}
\graphicspath{{fig/}{./figures/}}

% http://tex.stackexchange.com/a/542
% http://tex.stackexchange.com/q/204410
% Must go before float
%\usepackage{fixltx2e}

% Package which lets figures be inline 
\usepackage{float} 

% 2018-06-06
% Package for custom nested bullet-points/lists and inline variants
\usepackage[inline]{enumitem}

% Caption environment
\usepackage{caption}

% 2018-02-27
% Dirty workaround. `polyglossia` and `newclude` both define \providelength and
% \providecounter macros causing a compile error.
%https://tex.stackexchange.com/questions/297110/difficulties-compiling-document-possibly-related-to-moredefs
%http://tug.org/pipermail/xetex/2016-February/026384.html
\let\providelength\relax
\let\providecounter\relax

% Lets you use \include*{} to avoid the page breaks \include{} normally generates
\usepackage{newclude} 
% Lets you format ordinal numbers (2nd, 3rd, 4th etc.) in superscript
\usepackage{nth}

% Table of Contents
\usepackage{tocloft} 
\renewcommand{\cftsecleader}{\cftdotfill{\cftdotsep}}
\setcounter{tocdepth}{3}
\setcounter{secnumdepth}{3}

% Use international standard format for dates.
\usepackage[english]{isodate}

\usepackage[
    backend=biber,
    backref=true,
    hyperref=true,
    isbn=true,
    %style=authoryear,
    %style=numeric-comp,
    %citestyle=apa,
    %apabackref=true,
    %style=ieee,
    style=ieee,
    %style=acm,
    %citestyle=acm,
    % ACM style: https://tex.stackexchange.com/q/172147
    %style=trad-abbrv,
    %firstinits=true,
    url=true,
    %doi=true,
    %eprint=true,
    %autolang=other,
    %natbib
]{biblatex}


\addbibresource{diku.bib}
%\addbibresource{vtdat.bib}
\DeclareLanguageMapping{english}{english-apa}
%\input{fixcitation}

% 2016-03-12
% http://tex.stackexchange.com/q/60921/69620
% The following defines the Wikipedia-like superscript citations
\DeclareCiteCommand{\supercite}[\mkbibsuperscript]
{\iffieldundef{prenote}
    {}
    {\BibliographyWarning{Ignoring prenote argument}}%
    \iffieldundef{postnote}
    {}
{\BibliographyWarning{Ignoring postnote argument}}}
{\usebibmacro{citeindex}%
\bibopenbracket\usebibmacro{cite}\bibclosebracket}
{\supercitedelim}
{}
%and this overwrites the normal cite:
%\let\cite=\supercite



% 2016-01-19
% Make a blank line between paragraphs
% http://tex.stackexchange.com/a/74173
%\usepackage[parfill]{parskip}


% Package for typesetting programs. Listings does not support fsharp, but a little modification goes a long way
%\usepackage{listings}
\usepackage{color}

%\input{lstconfig}


% 2015-10-20
% http://tex.stackexchange.com/a/142409

% freely scalable fonts
%\usepackage{fix-cm}
%
%\newlength{\eightytt}
%\newcommand{\testthewidth}{%
%  \fontsize{\dimen0}{0}\selectfont
%  \settowidth{\dimen2}{x}%
%  \ifdim 80\dimen2>\textwidth
%    \advance\dimen0 by -.1pt
%    \expandafter\testthewidth
%  \else
%    \global\eightytt\dimen0
%  \fi
%}
%
%\AtBeginDocument{%
%  \dimen0=\csname f@size\endcsname pt
%  \begingroup
%  \ttfamily
%  \testthewidth
%  \endgroup
%  \lstset{
%    columns=fullflexible,
%    basicstyle=\fontsize{\eightytt}{1.2\eightytt}\ttfamily
%  }%
%}

% To make markdown apparence: `command`
\newcommand{\md}[1]
{
\fcolorbox{gittergray}{gitterblue}{\color{gittergray}{\texttt{ #1 }}} 
}

\renewcommand\thesection{\arabic{section}}



% Remove 'Chapter' text from chapter headings
\usepackage{titlesec}
\titleformat{\chapter}{\huge\bfseries}{\thechapter.}{20pt}{\huge\bfseries}

% 2016-06-11
% Used for colouring API table
% --Einar
\usepackage{color, colortbl}
\definecolor{GET}{HTML}{E7F0F7}
\definecolor{POST}{HTML}{E7F6EC}
\definecolor{PUT}{HTML}{F9F2E9}
\definecolor{DELETE}{HTML}{F5E8E8}

\usepackage{tabularx}

% 2016-03-22
% Make next part counter 0
\setcounter{part}{-1}
%\setcounter{section}{-1}
%\setcounter{section}{-1}
% Set 'part' counter to use Arabic numerals,
% since Roman numerals does not have a '0' to show.
\renewcommand\thepart{\arabic{part}}
\pagenumbering{arabic}

% 2016-08-11
% Rotate individual pages to landscape mode
\usepackage{pdflscape}

% 2018-06-20
% New (unofficial) KU frontpage package
\usepackage[english, ku, titlepage]{ku-frontpage}
% These will be the title and author, as included when \maketitle is called.
\assignment{Group Project}
\title{BFAST: Breaks For Addition Season and Trend} %
\subtitle{NDAK14008U 2018/2019} %
\author{
Jakob Stokholm Bertelsen %
\&
Einar Rasmussen % <einar@di.ku.dk> ftc208 <0x791699C0FFEEC0DE>
}
\advisor{Supervisor: Cosmin Oancea}
\date{Deadline: Friday, \isodate{2019-11-02 23:59 CET}}
%\frontpageimage{figures/denning.jpg}


% Adds explicit numbering by \numberthis to align* environment.
\newcommand\numberthis{\addtocounter{equation}{1}\tag{\theequation}}

% 2016-09-11
% \Better vector notation
\let\dddot\relax
\let\ddddot\relax
%\usepackage{mathtools}
%\usepackage{esvect}
%\renewcommand*{\Vec}[1]{\vv{\mathbf{#1}}}%

% 2016-0-15
% Poisson distributions and pmf
%\DeclareMathOperator{\Poisson}{Poisson}
%\DeclareMathOperator{\Cov}{Cov}
%\DeclareMathOperator{\Std}{Std}
%\DeclareMathOperator{\Var}{Var}
%\DeclareMathOperator{\argmin}{argmin}
%\DeclareMathOperator{\gcd}{gcd}
\DeclareMathOperator{\lcm}{lcm}

%\definecolor{light-gray}{gray}{0.9}
%\lstset{language=C, numbers=none, frame=none, backgroundcolor=\color{light-gray} }


%\usepackage{siunitx}
%\sisetup{output-exponent-marker=\ensuremath{\mathrm{e}}}

% lscape.sty Produce landscape pages in a (mainly) portrait document.
\usepackage{lscape}

\usepackage{multirow}

\usepackage{tikz}

\usepackage{xcolor}
\usepackage[outputdir=build]{minted}
%\usemintedstyle{trac}
\usemintedstyle{tango}
%\setminted{fontsize=\small}


% 2017-12-17
% http://www.tex.ac.uk/FAQ-verbfile.html
% provides \verbatiminput{verb.txt}
\usepackage{verbatim}

% 2018-02-27
%\usepackage{tombstone} % Conflicts with package `sematic`.


% 2018-02-27
% https://tex.stackexchange.com/a/380808
% The fvextra package is loaded by minted, so you should load minted before csquotes
%\usepackage[style=danish,danish=guillemets,strict=true]{csquotes}
\usepackage[style=french,french=guillemets,strict=true]{csquotes}
%\usepackage[style=english,strict=true]{csquotes}
%\usepackage[style=english,english=british,strict=true]{csquotes}



% 2018-03-01
% Fix border
% https://tex.stackexchange.com/a/88384/69620 
\makeatletter
\renewenvironment{minted@colorbg}[1]{
\setlength{\fboxsep}{\z@}
\def\minted@bgcol{#1}
%\noindent
\begin{lrbox}{\minted@bgbox}
\begin{minipage}{\linewidth}}
{\end{minipage}
\end{lrbox}%
\colorbox{\minted@bgcol}{\usebox{\minted@bgbox}}}
\makeatother

\definecolor{LightGray}{HTML}{f8f9fa}

\newmintedfile[ehaskell]{haskell}{
    tabsize=8, 
    fontsize=\footnotesize, 
    frame=lines,
    framesep=5\fboxrule,
    framerule=1pt,
    bgcolor=LightGray,
    rulecolor=\color{gray!40},
    linenos
    %baselinestretch=1.2,
    %firstline=0, 
    %lastline=17
}

\newmintedfile[efsharp]{fsharp}{
    tabsize=8, 
    %fontsize=\footnotesize, 
    frame=lines,
    framesep=5\fboxrule,
    framerule=1pt,
    bgcolor=LightGray,
    rulecolor=\color{gray!40},
    linenos
    %baselinestretch=1.2,
    %firstline=0, 
    %lastline=17
}

\newmintedfile[eprolog]{prolog}{
    tabsize=8, 
    fontsize=\footnotesize, 
    frame=lines,
    framesep=5\fboxrule,
    framerule=1pt,
    bgcolor=LightGray,
    rulecolor=\color{gray!40},
    linenos
    %baselinestretch=1.2,
    %firstline=0, 
    %lastline=17
}

\newminted{prolog}{
    tabsize=8, 
    fontsize=\footnotesize, 
    frame=lines,
    framesep=5\fboxrule,
    framerule=1pt,
    bgcolor=LightGray,
    rulecolor=\color{gray!40},
    linenos
    %baselinestretch=1.2,
    %firstline=0, 
    %lastline=17
}

\newmintedfile[etext]{text}{
    tabsize=8, 
    fontsize=\footnotesize, 
    frame=lines,
    framesep=5\fboxrule,
    framerule=1pt,
    bgcolor=LightGray,
    rulecolor=\color{gray!40},
    linenos
    %baselinestretch=1.2,
    %firstline=0, 
    %lastline=17
}
\newminted{text}{
    tabsize=8, 
    fontsize=\footnotesize, 
    frame=lines,
    framesep=5\fboxrule,
    framerule=1pt,
    bgcolor=LightGray,
    rulecolor=\color{gray!40},
    linenos
    %baselinestretch=1.2,
    %firstline=0, 
    %lastline=17
}

\newmintedfile[ecuda]{cuda}{
    tabsize=8, 
    fontsize=\footnotesize, 
    frame=lines,
    framesep=5\fboxrule,
    framerule=1pt,
    bgcolor=LightGray,
    rulecolor=\color{gray!40},
    linenos
    %baselinestretch=1.2,
    %firstline=0, 
    %lastline=17
}

\newminted{bash}{
    tabsize=8, 
    fontsize=\footnotesize, 
    frame=lines,
    framesep=5\fboxrule,
    framerule=1pt,
    bgcolor=LightGray,
    rulecolor=\color{gray!40},
    linenos
    %baselinestretch=1.2,
    %firstline=0, 
    %lastlne=17
}

\newmintedfile{gas}{
    tabsize=8, 
    fontsize=\footnotesize, 
    frame=lines,
    framesep=5\fboxrule,
    framerule=1pt,
    bgcolor=LightGray,
    rulecolor=\color{gray!40},
    linenos
    %baselinestretch=1.2,
    %firstline=0, 
    %lastlne=17
}

% 2018-03-13
% Adds a new type of label for \autoref{lst:mysourcecode.c}
\providecommand*{\listingautorefname}{Source-Code Listing}
\def\listingautorefname{Source-Code Listing}


\newcounter{refs}
\makeatletter
\defbibenvironment{counter}
  {\setcounter{refs}{0}
  \renewcommand{\blx@driver}[1]{}
  }
  {We have \therefs references}
  {\stepcounter{refs}}
\makeatother









\usepackage{latexsym}
%\newcommand{\qed}{\rule{2mm}{2mm}}



\newtheorem*{etheorem}{Theorem}

% https://tex.stackexchange.com/questions/302255/what-is-the-symbol-of-end-of-a-proof
%\renewcommand{\qedsymbol}{\ensuremath{\blacksquare}}
\renewcommand{\qedsymbol}{\ensuremath{\rule{2mm}{2mm}}}


\DeclareMathOperator{\abs}{abs}

% https://tex.stackexchange.com/a/118217/69620
\DeclarePairedDelimiter\ceil{\lceil}{\rceil}
\DeclarePairedDelimiter\floor{\lfloor}{\rfloor}


% 2018-04-09
% Defines `\inference` environment used for Operational Semantics
\usepackage{semantic}

%\usepackage[export]{adjustbox}
%\usepackage{subfig}
\usepackage{subcaption}

\newmintedfile[uissql]{sql}{
    tabsize=8, 
    fontsize=\footnotesize, 
    frame=lines,
    framesep=5\fboxrule,
    framerule=1pt,
    %bgcolor=LightGray,
    %rulecolor=\color{gray!40},
    linenos
}


%\usepackage[toc,page,title,titletoc,header]{appendix}

\usepackage{multicol}



%\setlength{\parindent}{1cm} % Default is 15pt.
\setlength{\parindent}{15pt} % Default is 15pt.

\usepackage{nicefrac}
