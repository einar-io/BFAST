\subsection{
    Step 7b: \texttt{BOUNDS}
}

This step is translated from \autoref{fut:kernel7b} into \autoref{cuda:kernel7b}.
Notice that this kernel does not
depend on any input. It produces an output array of length \texttt{N-n}
approximately around 186. Initially we decided to do it in host code, arguing we
would not save much by parallellize it. It was the kernel with the shortest
running time of 9.80 µs, and thus the least opportunity.

However, since we needed the output to reside in device memory for the later
kernels, we chose to move it to device code in the end. This has actually caused
towards a 200\% slowdown to around 29.12 µs for some measurements.  But since
this negible, we chose to keep in on the device.  The launch code is seen in
\autoref{cuda:kernel7b_run}.


\begin{figure}[H]
    \centering
    \ehaskell[firstline=191,lastline=194]{../src/fut-handout/bfast-distrib.fut}
    \caption{Futhark function for calculating \texttt{BOUND}.}
    \label{fut:kernel7b}
\end{figure}

\begin{figure}[H]
    \centering
    \ecuda[firstline=327,lastline=345]{../src/kernels/bfast_others.cu}
    \caption{CUDA kernel 7b for calculating  \texttt{BOUND}.}
    \label{cuda:kernel7b}
\end{figure}

\begin{figure}[H]
    \centering
    \ecuda[firstline=347,lastline=356]{../src/kernels/bfast_others.cu}
    \caption{CUDA launch syntax for kernel 7.}
    \label{cuda:kernel7b_run}
\end{figure}

