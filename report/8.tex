\subsection{
    Step 8: Calculating Breaks
}

Kernel 8 from \autoref{fut:kernel8} seen in \autoref{cuda:kernel8} produces the
resulting array of vectors of \textsc{BFAST}. 
Each vector contains the break values for each pixel in the monitor period.

On line 46 we copy \texttt{BOUND} into shared memory. This gave us a small, but
consistent, speedup from 1747.46 µs to 1743.30 µs possibly due to coalesced
reads. %LINUM

Line 5 keeps the return value of
\texttt{scaninc\_block\_add\_nowrite<float>(MO\_shr)} in a register in order to
decrease the amount of unneeded memory accesses.%
%unneeded usage of memory bandwidth. %LINUM


\begin{figure}[H]
    \centering
    \ehaskell[firstline=197,lastline=217]{../src/fut-handout/bfast-distrib.fut}
    \caption{Futhark function for calculating \texttt{breakss}.}
    \label{fut:kernel8}
\end{figure}

\begin{figure}[H]
    \centering
    \ecuda[firstline=2,lastline=73]{../src/kernels/bfast_step_8.cu}
    \caption{CUDA kernel for calculating  \texttt{breakss}.}
    \label{cuda:kernel8}
\end{figure}


It seems intuitive that line 58---63 of \autoref{cuda:kernel8} %LINUM
write either the value of \texttt{val} or \texttt{NAN} to the memory location in
line 62. However, in order to do so, we must show that no index of \texttt{MO\_shr} 
can occur twice and that no index is never written, and thus we conservatively follow the
control flow of \autoref{fut:kernel8} and initially write \texttt{NAN} in every
index of the array. 
The impact was confirmed to be negligible due to the lockstep execution
model of GPUs:
 We created an experimental version called \texttt{bfast\_step\_8\_opt} and seen
 in \autoref{cuda:kernel8alt} which also validates. However, the running time of
 \texttt{bfast\_step\_8\_reuse} is 1745.06 µs compared to \texttt{bfast\_step\_8\_opt2} of
 1744.33 µs and thus we did not pursue to prove the former correct.

\begin{figure}[H]
    \centering
    \ecuda[firstline=181,lastline=186]{../src/kernels/bfast_step_8.cu}
    \caption{Alternative control flow for kernel 8.}
    \label{cuda:kernel8alt}
\end{figure}



